\chapter{Considerations for future research:}

The present analysis aimed to understand the imposed impacts of \ac{oae} on the seasonal cycle of \ch{CO2} flux and ocean \ch{pCO2} in European waters, and how such changes vary depending on the future background climate. Given the knowledge vacuum present in current literature, these results were able to provide an in-depth representation of the mechanisms triggered by \ac{oae}, together with a freely accessible GitHub repository that supports open-source research. Although defining precise interpretations on the observed system changes is beyond the scope of this study, the results presented above can help formulate some key conclusions for future investigation. 

To answer the first research question addressing the impacts of \ac{oae} on the seasonal cycle of ocean \ch{pCO2} and \ch{CO2} flux, \ac{oae} encourages a compression of the former and an amplification of the latter. Secondly, to explain the causes and effects of such variations, it can be concluded that the ocean \ch{pCO2} compression is the result of its reduced sensitivity to \ac{dic} inputs. This mechanism allows for a greater \ch{CO2} modulation given by the larger air-to-sea gas disequilibrium, hence the \ch{CO2} amplification pattern. Lastly, the higher climate background state (SSP3-7.0) forces a magnification of the aforementioned \ac{oae}-induced effects. Alkalinity and \ac{dic} variations are also more pronounced under SSP3-7.0, although ocean acidification is mitigated more efficiently under SSP1-2.6. 

Drawing from this study's results, theoretical assumptions can be made on both the ideal location and the desirable time to perform \ac{oae}. From existing literature, it is established that fast air-sea re-equilibration happens in conditions of shallow \ac{mld} and strong wind-driven mixing \citep{jones2014spatial}. Geographically, regions like the southern North Sea represent a favourable location as they display a shallow water column, which impedes alkalinity loss to the subsurface, and which, especially in wintertime, undergo strong wind-induced mixing, speeding up the adjustment.

Due to the asymmetrical uptake change that \ac{oae} generated in the experiments, and assuming two to three-month equilibration rate in areas like the southern North Sea, the best time to perform \ac{oae} would be in the season previous to the phase that already produces highest ocean \ch{CO2} uptake. Taking the southern North Sea as an example, alkalinity should be added in autumn to allow for full equilibration over winter and maximise \ac{oae} potential. This conclusion, however, disagrees with \cite{lenton2018assessing}, who did not observe any seasonally-dependent efficiency. 

Given these conclusions, some limitations and recommendations for future research can be discussed. Firstly, although the European coastline accommodates multiple river mouths, riverine inputs of alkalinity are not resolved in \ac{foci}. This creates a vacuum for generating realistic data on alkalinity and on its dependent variables. Likewise, coastal sediment biogeochemistry is poorly resolved in the model, although alkalinity generation from the anaerobic degradation of \ac{om} is a major process that regulates seasonality in the region of focus \citep{thomas2009enhanced}. It would therefore be valuable to integrate these components in the model and assess more realistic data for future scientific assumptions. 

In past literature \citep{schwinger2022report, fassbender2022quantifying}, \ch{CO2} seasonality was investigated by separating the thermal (\ac{sst}) from the non-thermal (\ac{dic}) component. The same approach could be applied in this case to weigh the relative impact of each on the seasonal \ch{CO2} cycle. Additionally, it is well established that seasonal \ch{CO2} fluxes are influenced by, among others, wind speed \citep{jones2014spatial} and \ac{mld} \citep{jo2022future, jones2014spatial}. Evaluating the part that these components play in regulating the seasonality of \ch{CO2} would therefore benefit the present study and give insight on the possible loops triggered by \ac{oae} application.

Moreover, as this analysis is carried out for the first 50 metres of the ocean surface, it is not possible to see where accumulated \ac{dic} travels beyond this threshold. Investigating the pathway that \ac{dic} would take and whether, for example, it would spread far from where it sinks or whether it would resurface and be respired beyond the model domain are all relevant assumptions that should be tested. On this note, given the influence of the \ac{mld} in the seasonal carbon cycle, replacing the 50-metre threshold with the calculated \ac{mld} (of which a \ac{foci} variable already exists) would better describe water transport as a region of uniform mixing. 

Regarding the features of this study's experiments, they cover only one of the many ways in which \ac{oae} could be implemented. Other, practically easier schemes are already under investigation: addition performed in seasonal pulses, rather than continuously, or at defined point sources such as river mouths, rather than along an entire coastline. Different application simulations can help understand under which conditions alkalinity enhancement is most effective.

In conclusion, this study contributed to shed light on the impacts of \ac{oae} on the carbon cycle by analysing its seasonal fluxes. As climate change forces global temperatures to rise, the ocean will likely suffer from a more precarious equilibrium and will be more vulnerable to thermally-driven changes. With the theoretical potential of \ac{oae} to alleviate climate warming, further investigation is encouraged and may create the premises for conscious large-scale application. 